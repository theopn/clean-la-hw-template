%%%%%%%%%%%%%%%%%%%%%%%%%%%%%%%%%%%%%%%%%%%%%%%%%%%%%%%%%%%%%%%%%%%%%%%%%%%%%%%%%%%%%%%%%%%%%%%%%%%%
% Author: Theo Park
% Description: A simple and cLeAn hw template
%              mainly for lin alg hw
% Also this template is supposed to be portable
% Inspired  by: https://github.com/jdavis/latex-homework-template,
%               https://gist.github.com/marethyu/
% Why?: Why not
% Last Updated: 13 April 2022
% Good luck w ur homework
%%%%%%%%%%%%%%%%%%%%%%%%%%%%%%%%%%%%%%%%%%%%%%%%%%%%%%%%%%%%%%%%%%%%%%%%%%%%%%%%%%%%%%%%%%%%%%%%%%%%

%%%%%%%%%% Type Homework Information in {}!! %%%%%%%%%%%
\newcommand{\homework}{\textbf{Homework \#332}}
\newcommand{\name}{Theo Park}
\newcommand{\class}{MA687}
\newcommand{\prof}{\textit{Prof. Carl Gauss}}
\newcommand{\dueDate}{Due on: 32 April 1954}
%%%%%%%%%% ----------------------------- %%%%%%%%%%%

%% Basic set ups and pkg import %%
\documentclass{article}

\usepackage[left=2cm, right=2cm, top=3cm, bottom=3cm]{geometry}
\usepackage[plain]{algorithm}
\usepackage{amsmath}
\usepackage{amsthm}
\usepackage{amsfonts}
\usepackage{amssymb}
\usepackage{extramarks}
% My fav font %
\usepackage{palatino}
% Header settings %
\usepackage{fancyhdr}
\lhead{\name}
\chead{\class : \homework}
\rhead{\firstxmark}
\lfoot{\lastxmark}
\cfoot{\thepage}

\linespread{1.1}
\pagestyle{fancy}
%% --------------------------- %%

%%%%%%%%%% Custom commands related to template itself %%%%%%%%%%

%%%%%% This part is straight from jdavis template %%%%%
%%%%%% (except deleting set counter and adding pagebreak), %%%%%
%%%%%% and I am planning on changing it in the future %%%%%
\newcommand{\enterProblemHeader}[1]{
	\nobreak\extramarks{}{Problem \arabic{#1} continued on next page\ldots}\nobreak{}
	\nobreak\extramarks{Problem \arabic{#1} (continued)}{Problem \arabic{#1} continued on next page\ldots}\nobreak{}
}

\newcommand{\exitProblemHeader}[1]{
	\nobreak\extramarks{Problem \arabic{#1} (continued)}{Problem \arabic{#1} continued on next page\ldots}\nobreak{}
	\stepcounter{#1}
	\nobreak\extramarks{Problem \arabic{#1}}{}\nobreak{}
}
\setcounter{secnumdepth}{0}
\newcounter{homeworkProblemCounter}
\setcounter{homeworkProblemCounter}{1}
\nobreak\extramarks{Problem \arabic{homeworkProblemCounter}}{}\nobreak{}
\newenvironment{homeworkProblem}[1][-1]{
	\ifnum#1>0
	\setcounter{homeworkProblemCounter}{#1}
	\fi
	\section{Problem \arabic{homeworkProblemCounter}}
	\enterProblemHeader{homeworkProblemCounter}
}{
	\exitProblemHeader{homeworkProblemCounter}
  \pagebreak
}

% Question environment
\newenvironment{question}{
}

% Solution commands; use NoNewLine for underfull hbox err %
\newcommand{\solution}{\subsection{\large Solution}}
\newcommand{\solutionNoNewLine}{\ \\ \textbf{\large Solution}}

% Homework with points %
\newcommand{\points}[1]{\vspace*{-10pt}\paragraph*{{(#1 Points)}}}
%%%%%% ---------------------------------------------------------------- %%%%%

%%%%%%%%%% ------------------------------------------ %%%%%%%%%%

%%%%%%%%%% Custom commands related to math %%%%%%%%%%
% Elementary row operation arrow
% ex: \arrow3{R_1 - R_2}{}{}
\newcount\arrowcount
\newcommand\arrows[1]{
        \global\arrowcount#1
        \ifnum\arrowcount>0
                \begin{matrix}
                \expandafter\nextarrow
        \fi
}

\newcommand\nextarrow[1]{
        \global\advance\arrowcount-1
        \ifx\relax#1\relax\else \xrightarrow{#1}\fi
        \ifnum\arrowcount=0
                \end{matrix}
        \else
                \\
                \expandafter\nextarrow
        \fi
}

% Augmented matrix command
\newenvironment{amatrix}[1]{%
  \left[\begin{array}{@{}*{#1}{c}|c@{}}
}{%
  \end{array}\right]
}

% Augmented matrix when you find an inverse
% e.g \begin{invAmatrix}{2} creates side by side 2 X 2 matrix
\newenvironment{invAmatrix}[1]{%
  \left[\begin{array}{@{}*{#1}{c}|*{#1}{c}@{}}
}{%
  \end{array}\right]
}
%%%%%%%%%% ------------------------------- %%%%%%%%%%

%%%%%%%%%%%%%%%%%%%%%%%%%%%%%%%%%%%%%%%%%%%%%%%%%%
\begin{document}
    %%%%% Title page %%%%%
    \thispagestyle{plain}
    \begin{center}
        \vspace*{150pt}
        \LARGE{\homework}\\
        \Large{\name}\\
        \vspace*{150pt}
        \Large{\class - \prof}\\
        \vspace{10pt}
        \large{\dueDate}
    \end{center}
    \pagebreak
    %%%%% ---------- %%%%%

    % Homework 1 - My plan is to take homework # as an argument
    \begin{homeworkProblem}
        \begin{question}
            Hi students my name is Carl and use Gaussian elemination to find solution for
            \begin{align*}
                \begin{amatrix}{3}
                    1 & 1 & 1 & 3\\
                    2 & 3 & 7 & 0\\
                    1 & 3 & -2 & 17
                \end{amatrix}
            \end{align*}
            I have no clue why they named it after me when I literally just copied and pasted from some random Asian math book
        \end{question}
        % I will fix points underfull issue
        \points{23}

        \solution
        Hello so this is my solution\\
        $\begin{amatrix}{3}
          1 & 1 & 1 & 3\\
          2 & 3 & 7 & 0\\
          1 & 3 & -2 & 17
        \end{amatrix}
        \arrows3{}{R_2 - 2R_1}{R_3 - R_1}
        \begin{amatrix}{3}
          1 & 1 & 1 & 3\\
          0 & 1 & 5 & -7\\
          0 & 2 & -3 & 14 
        \end{amatrix}
        \arrows3{}{}{R_3 - 2R_2}
        \begin{amatrix}{3}
          1 & 1 & 1 & 3\\
          0 & 1 & 5 & -7\\
          0 & 0 & -13 & 26  
        \end{amatrix}
        \arrows3{}{}{\frac{R_3}{-13}}
        \begin{bmatrix}
          1 & 1 & 1 & 3\\
          0 & 1 & 5 & -7\\
          0 & 0 & 1 & -2 
        \end{bmatrix}$\\
        $\begin{cases}
          x + y + z = 3\\
          y + 5z = -6\\
          z = -2
        \end{cases} 
        \therefore \begin{cases}
          x = 1\\
          y = 4\\
          z = -2
        \end{cases}$
        Hello so this is my solution\\
        $\begin{amatrix}{3}
          1 & 1 & 1 & 3\\
          2 & 3 & 7 & 0\\
          1 & 3 & -2 & 17
        \end{amatrix}
        \arrows3{}{R_2 - 2R_1}{R_3 - R_1}
        \begin{amatrix}{3}
          1 & 1 & 1 & 3\\
          0 & 1 & 5 & -7\\
          0 & 2 & -3 & 14 
        \end{amatrix}
        \arrows3{}{}{R_3 - 2R_2}
        \begin{amatrix}{3}
          1 & 1 & 1 & 3\\
          0 & 1 & 5 & -7\\
          0 & 0 & -13 & 26  
        \end{amatrix}
        \arrows3{}{}{\frac{R_3}{-13}}
        \begin{bmatrix}
          1 & 1 & 1 & 3\\
          0 & 1 & 5 & -7\\
          0 & 0 & 1 & -2 
        \end{bmatrix}$\\
        $\begin{cases}
          x + y + z = 3\\
          y + 5z = -6\\
          z = -2
        \end{cases} 
        \therefore \begin{cases}
          x = 1\\
          y = 4\\
          z = -2
        \end{cases}$
        Hello so this is my solution\\
        $\begin{amatrix}{3}
          1 & 1 & 1 & 3\\
          2 & 3 & 7 & 0\\
          1 & 3 & -2 & 17
        \end{amatrix}
        \arrows3{}{R_2 - 2R_1}{R_3 - R_1}
        \begin{amatrix}{3}
          1 & 1 & 1 & 3\\
          0 & 1 & 5 & -7\\
          0 & 2 & -3 & 14 
        \end{amatrix}
        \arrows3{}{}{R_3 - 2R_2}
        \begin{amatrix}{3}
          1 & 1 & 1 & 3\\
          0 & 1 & 5 & -7\\
          0 & 0 & -13 & 26  
        \end{amatrix}
        \arrows3{}{}{\frac{R_3}{-13}}
        \begin{bmatrix}
          1 & 1 & 1 & 3\\
          0 & 1 & 5 & -7\\
          0 & 0 & 1 & -2 
        \end{bmatrix}$\\
        $\begin{cases}
          x + y + z = 3\\
          y + 5z = -6\\
          z = -2
        \end{cases} 
        \therefore \begin{cases}
          x = 1\\
          y = 4\\
          z = -2
        \end{cases}$
        Hello so this is my solution\\
        $\begin{amatrix}{3}
          1 & 1 & 1 & 3\\
          2 & 3 & 7 & 0\\
          1 & 3 & -2 & 17
        \end{amatrix}
        \arrows3{}{R_2 - 2R_1}{R_3 - R_1}
        \begin{amatrix}{3}
          1 & 1 & 1 & 3\\
          0 & 1 & 5 & -7\\
          0 & 2 & -3 & 14 
        \end{amatrix}
        \arrows3{}{}{R_3 - 2R_2}
        \begin{amatrix}{3}
          1 & 1 & 1 & 3\\
          0 & 1 & 5 & -7\\
          0 & 0 & -13 & 26  
        \end{amatrix}
        \arrows3{}{}{\frac{R_3}{-13}}
        \begin{bmatrix}
          1 & 1 & 1 & 3\\
          0 & 1 & 5 & -7\\
          0 & 0 & 1 & -2 
        \end{bmatrix}$\\
        $\begin{cases}
          x + y + z = 3\\
          y + 5z = -6\\
          z = -2
        \end{cases} 
        \therefore \begin{cases}
          x = 1\\
          y = 4\\
          z = -2
        \end{cases}$
                Hello so this is my solution\\
        $\begin{amatrix}{3}
          1 & 1 & 1 & 3\\
          2 & 3 & 7 & 0\\
          1 & 3 & -2 & 17
        \end{amatrix}
        \arrows3{}{R_2 - 2R_1}{R_3 - R_1}
        \begin{amatrix}{3}
          1 & 1 & 1 & 3\\
          0 & 1 & 5 & -7\\
          0 & 2 & -3 & 14 
        \end{amatrix}
        \arrows3{}{}{R_3 - 2R_2}
        \begin{amatrix}{3}
          1 & 1 & 1 & 3\\
          0 & 1 & 5 & -7\\
          0 & 0 & -13 & 26  
        \end{amatrix}
        \arrows3{}{}{\frac{R_3}{-13}}
        \begin{bmatrix}
          1 & 1 & 1 & 3\\
          0 & 1 & 5 & -7\\
          0 & 0 & 1 & -2 
        \end{bmatrix}$\\
        $\begin{cases}
          x + y + z = 3\\
          y + 5z = -6\\
          z = -2
        \end{cases} 
        \therefore \begin{cases}
          x = 1\\
          y = 4\\
          z = -2
        \end{cases}$
    \end{homeworkProblem}

    \begin{homeworkProblem}
      \begin{question}
        That was easy right? Because I am a good professor who assigns only 2 question per homework, here's the final questions.
        \begin{itemize}
          \item Find the inverse of 
        \end{itemize}
      \end{question}
    \end{homeworkProblem}
\end{document}
%%%%%%%%%%%%%%%%%%%%%%%%%%%%%%%%%%%%%%%%%%%%%%%%%%